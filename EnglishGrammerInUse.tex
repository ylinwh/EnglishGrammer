\documentclass[12pt]{article}
\usepackage{parskip}    % disable autoindent
\usepackage{enumitem}   % customize the style of the counters
\usepackage{geometry}
\geometry{letterpaper, portrait, margin=1in}

\begin{document}
    \section{Present continuous(I am)}
    \textbf{I am doing something}=I'm in the middle of doing it; I've started doing it and I haven't
    finished. 

    Sometimes the action is not happening at the time of speaking. For example: \\
    Steve is talking to a friend on the phone. He says: "I'm reading a really good book at the moment.
    It's about a man who ..."

    Steve is not reading the book at the time of speaking. He means that he started it, but has not finished it yet.
    He is in the middle of reading it.

    You can use the present continuous with \textbf{today} / \textbf{this week} / \textbf{this year} etc. (periods around now):
    \begin{enumerate}
        \item Here is a conversation:
        \begin{itemize}
            \item A: You're working hard today. (not You work hard today)
            \item B: Yes, I have a lot to do.
        \end{itemize}
        \item The company I work for isn't doing so well \textbf{this year}.
        \item I want to lose weight, so this week I am not eating lunch.
    \end{enumerate}

    We use the present continuous when we talk about changes happening around now, especially with these verbs:\\
    \textbf{get} \quad \textbf{change} \quad \textbf{become} \quad \textbf{increase} \quad \textbf{rise} \quad \textbf{fall} \quad \textbf{grow}
    \quad \textbf{improve} \quad \textbf{begin} \quad \textbf{start}

    \section{Present simple (I do)}
    We use the present simple to talk about things in general. We use it to say that something happens all the time or repeatdly,
    or that something is true in general.

    We use the present simple to say how often we do things:
    \begin{enumerate}
        \item I get up at 8 o'clock every morning.
        \item How often do you go to the dentist?
    \end{enumerate}

    Sometimes we do things by saying something. For example, when you promise to do something, 
    you can say '\textbf{I promise ...}'; when you suggest something, you can say '\textbf{I suggest ...}':
    \begin{enumerate}
        \item I promise I won't be late. (not I'm promising)
        \item 'What do you suggest I do?' 'I suggest that you ...'
    \end{enumerate}
    In the same way we say: \textbf{I apologise} ... / \textbf{I advise} ... / \textbf{I inist} ... / \textbf{I agree} ... /
    \textbf{I refuse} ... etc.

    \section{Present continuous and present simple 1 (I am doing and I do)}
    \textit{Present continuous} (\textbf{I am doing}) 

    We use the continuous for things happening at or around the time of speaking. The action is not complete.\\

    \textit{Present simple} (\textbf{I do})

    We use the simple for things in general or things that happens repeatdly.

    \textbf{Note:} I always do and I'm always doing

    \begin{enumerate}
        \item I always do (something) = I do it every time
        \begin{itemize}
            \item I always go to work by car.
        \end{itemize}
        \item I'm always doing something: (perhaps something is more often than normal)
        \begin{itemize}
            \item I'm always losing things.
            \item You're always playing computer games. You should do something more active. (=You play computer games too often.)
        \end{itemize}
    \end{enumerate}

    \section{Present continuous and present simple 2 (I am doing and I do)}
    We use continuous forms for action and happening that have started but not finished (they are eating / it is raining etc.). Some
    verbs (for example, known and like) are not normal used in this way. We don't say 'I am knowning' or 'they are liking'; we say 'I know', 'they like'.

    The following verbs are not normal used in the present continuous: \\
    \textbf{like \quad want \quad need \quad prefer \quad know \quad realise \quad suppose \quad mean \quad understand \quad believe
    \quad remember \quad belong \quad fit \quad contain \quad consist \quad seem}

    Examples:
    \begin{enumerate}
        \item Think
        \begin{itemize}
            \item I think Mary is Canadian, but I'm not sure.
            \item What do you think of my plan?
            \item I'm thinking about what happend. I often think about it. (When think means 'consider', the continuous is possible.)
            \item Nicky is thinking of giving up her job. (=she is considering it.)
        \end{itemize}
        
        \item see hear smell taste
        \begin{itemize}
            \item Do you see that man over there? (not Are you seeing)
            \item I can hear a strange noise. Can you hear it? (we often use can + see/hear/smell/taste)
        \end{itemize}
        \item look fell
        \begin{itemize}
            \item You look well today. or You're looking well today. (you can use the present simple or continuous to say how somebody
            looks or feels now)
            \item How do you feel now? or How are you feeling now ?
            \item  I usually feel tired in the morning. (not I'm usually feeling)
        \end{itemize}

        \item He is selfish and He is being selfish
        \begin{itemize}
            \item He's being = He's behaving / He's acting.
            \item I can't understand why he's being so selfish. He isn't usually like that. (being self = behaving selfish at the moment)

            \item (Note:) We use am/is/are being to say how somebody is behaving. it is not usually possible in other sentences
            \item It's hot today. (not It is being hot)
            \item Sarah is very tired. (not is being tired)
        \end{itemize}
    \end{enumerate}
    
    \section{Past simple (I did)}
    Very often the past simple ends in -\textbf{ed} (regular verbs). But many verbs are \textit{irregular}.
    The past simple does not end in -\textbf{ed}.\\
    Write -- wrote \quad see -- saw \quad sell -- sold \quad throw -- threw \quad catch -- caught

    \section{Past continuous (I was doing)}
    Yesterday Karen and Jim played tennis. They started at 10 o'clock and finished at 11.30.\\
    So, at 10.30 they \textbf{were playing} tennis.\\

    They \textbf{were playing} = they were in the middle of playing. They had not finished playing.\\

    \textbf{Compared the past continuous (I was doing) and past simple (I did):}
    \begin{enumerate}
        \item Past continuous (in the middle of an action)
        \begin{itemize}
            \item I was walking home when I met Dan. (in the middle of walking home)
            \item Kate was watching TV when we arrived.
        \end{itemize}

        \item Past simple (complete action)
        \begin{itemize}
            \item I walked home after party last night. (=all the way, completely)
            \item Kate watched television a lot when she was ill last year.
        \end{itemize}
    \end{enumerate}

    \textbf{Note:} We often use the past simple and the past continuous together to say that something happend
    in the middle of something else.
    \begin{enumerate}
        \item Matt phoned while we were having dinner.
        \item I saw you in the park yesterday. You were sitting on the grass and reading a book.
    \end{enumerate}

    But, we use the past simple to say that one thing happend after another.
    \begin{enumerate}
        \item I was walking along the road when I saw Dan. So I stopped, and we had a chat.
        \item Compare
        \begin{itemize}
            \item When Karen arrived, we were having dinner. (=we had already started before she arrived.)
            \item When Karen arrived, we had dinner. (=Karen arrived, and then we had dinner)
        \end{itemize}
    \end{enumerate}

    Some verbs (e.g. \textbf{know} and \textbf{want}) are not normally used in the continuous(see Unit 4A):
    \begin{enumerate}
        \item We were good friends. We \textbf{knew} each other well. (not We were knowing)
        \item I was enjoying the party, but Chris \textbf{wanted} to go home. (not was wanting)
    \end{enumerate}
\end{document}