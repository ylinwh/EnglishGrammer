\documentclass[12pt]{article}
\usepackage{parskip}    % disable autoindent
\usepackage{enumitem}   % customize the style of the counters
\begin{document}
    \section{Present continuous(I am)}
    \textbf{I am doing something}=I'm in the middle of doing it; I've started doing it and I haven't
    finished. 

    Sometimes the action is not happening at the time of speaking. For example: \\
    Steve is talking to a friend on the phone. He says: "I'm reading a really good book at the moment.
    It's about a man who ..."

    Steve is not reading the book at the time of speaking. He means that he started it, but has not finished it yet.
    He is in the middle of reading it.

    You can use the present continuous with \textbf{today} / \textbf{this week} / \textbf{this year} etc. (periods around now):
    \begin{enumerate}
        \item Here is a conversation:
        \begin{itemize}
            \item A: You're working hard today. (not You work hard today)
            \item B: Yes, I have a lot to do.
        \end{itemize}
        \item The company I work for isn't doing so well \textbf{this year}.
        \item I want to lose weight, so this week I am not eating lunch.
    \end{enumerate}

    We use the present continuous when we talk about changes happening around now, especially with these verbs:\\
    \textbf{get} \quad \textbf{change} \quad \textbf{become} \quad \textbf{increase} \quad \textbf{rise} \quad \textbf{fall} \quad \textbf{grow}
    \quad \textbf{improve} \quad \textbf{begin} \quad \textbf{start}

    \section{Present simple (I do)}
    We use the present simple to talk about things in general. We use it to say that something happens all the time or repeatdly,
    or that something is true in general.

    We use the present simple to say how often we do things:
    \begin{enumerate}
        \item I get up at 8 o'clock every morning.
        \item How often do you go to the dentist?
    \end{enumerate}

    Sometimes we do things by saying something. For example, when you promise to do something, 
    you can say '\textbf{I promise ...}'; when you suggest something, you can say '\textbf{I suggest ...}':
    \begin{enumerate}
        \item I promise I won't be late. (not I'm promising)
        \item 'What do you suggest I do?' 'I suggest that you ...'
    \end{enumerate}
    In the same way we say: \textbf{I apologise} ... / \textbf{I advise} ... / \textbf{I inist} ... / \textbf{I agree} ... /
    \textbf{I refuse} ... etc.
\end{document}